% This page defines the format for Vitas

\font\tentt=cmtt10
\font\sltt=cmsltt10
\let\mainfont=\tenrm

\font\tenssb=cmssbx10
\font\twelvess=cmss10 at 12pt
\font\ninerm=cmr9
\font\eightrm=cmr8
\font\eightsl=cmsl8
\def\sc{\eightrm}

\def\TeX{T\hbox{\hskip-.1667em\lower.424ex\hbox{E}\hskip-.125em X}}
\font\logo=manfnt % font used for the METAFONT logo
\def\MF{{\logo META}\-{\logo FONT}\spacefactor1000\relax}
\def\slMF{{\logo 89:;}\-{\logo <=>:}} % slanted version
\let\,=\thinspace
\def\?{\mkern-1mu}
\chardef\_=`\_

\def\og#1{\leavevmode\smash{\vtop{% crude approximation of Polish ogonek
  \baselineskip0pt\lineskip0pt\lineskiplimit0pt
  \ialign{##\crcr\relax#1\cr
    \hidewidth\kern.2em
    \dimen0=.0040ex \multiply\dimen0\fontdimen1\font
    \kern-.0156\dimen0`\hidewidth\cr}}}}

\def\p #1.{\vskip 3pt plus 2pt minus 1pt
 \noindent\hbox to 20pt{\hss\bf#1. }\hangindent 20pt\ignorespaces}

\outer\def\finishpage{\par\vfill\eject}

\def\\{\par\hangindent 20pt\noindent\ignorespaces}

\outer\def\sectionbegin #1. #2{\vskip 0pt plus 400pt\penalty-100
 \vskip 12pt plus -394pt minus 4pt
 \noindent\tenssb\hbox to 0pt{\hss#1. }#2
 \par\penalty10000\vskip 3pt plus 2pt minus 1pt\rm}

\def\pat#1 {\par\noindent
 \hbox to 40pt{\hss #1\hskip 5pt}\hangindent 40pt\ignorespaces}

\def\yskip {\vskip 3pt plus 2pt minus 1pt}
\def\yyskip {\penalty-100\vskip 6pt plus 4pt minus 2pt}
\def\xskip {\hskip 7pt plus 3pt minus 4pt}

\def\star{\hbox to 0pt{\hss *}}

\newdimen\unicodeptsize \unicodeptsize=10pt

\begingroup
\catcode`@=11 % we will define two private macro names

\gdef\Uni#1:#2:#3:#4:#5<#6>% ems:cols:rows:-hoff:rows+voff<hexbitmap>
 {\leavevmode \hbox to#1\unicodeptsize
    {\special{" 0 0 moveto currentpoint translate
                \unic@deptsize \unic@deptsize scale #2 #3 true
                [24 0 0 -24 #4 #5] {<#6>} imagemask}\hss}}
\gdef\NUni#1:#2:#3:#4:#5<#6>% ems:cols:rows:-hoff:rows+voff<hexbitmap>
 {\leavevmode \hbox to#1\unicodeptsize
    {\special{" 0 0 moveto currentpoint translate
                \unic@deptsize \unic@deptsize scale #2 #3 true
                [74 0 0 -74 #4 #5] {<#6>} imagemask}\hss}}

{\catcode`p=12\catcode`t=12\gdef\uni@ff#1pt{#1}}
\gdef\unic@deptsize{\expandafter\uni@ff\the\unicodeptsize\space}

\endgroup

\def\today{\ifcase\month\or
  January\or February\or March\or April\or May\or June\or
  July\or August\or September\or October\or November\or December\fi
  \space\number\day, \number\year}

%%%%%

\line{\twelvess CURRICULUM VIT\kern-1.5pt\AE\hfill\eightrm\today}
\vskip 10pt

\sectionbegin 1. {Biographical and Personal Information}
\\Ole P. Orhagen, born August 21, 1989, Gj{\o}vik, Oppland, Norway; Norwegian citizen.
\sectionbegin 2. {Academic History}
\\University of the Sunshine Coast, September 2010--June 2014; B.S., Major English, Minor Mathematics.
\\University of Oslo, September 2014--May 2019; B.S. Computer Science, M.Sc. in
Cyber\-n\-e\-tics, May 2019. Thesis: ``The RRT Funnel Algorithm.''

\sectionbegin 3. {Employment Record}
\\Teacher's Assistant, University of Oslo, Oslo, 2015--2016.
\\Intern, Northern.tech, Oslo, 2016--2019.
\\Junior Software Engineer, Northern.tech, Oslo, 2019--2021.
\\Software Engineer, Northern.tech, Oslo, 2021--2022.
\\Senior Software Engineer, Northern.tech, Oslo, 2022--

\sectionbegin 4. {Publications}

\\ \star Orhagen, Ole Petter. The development of the RRT-Funnel motion planning algorithm. MS thesis. 2019, link: {\tt https://www.duo.uio.no/bitstream/handle/10852/69325/1/olepor\_thesis\_2019.pdf}.
\\ \star Orhagen, Ole Petter, Marius Thoresen, and Kim Mathiassen. "The rapidly exploring random tree funnel algorithm." 2022 8th International Conference on Mechatronics and Robotics Engineering (ICMRE). IEEE, 2022, link: {\tt https://ieeexplore.ieee.org/abstract/document/9734089}.


\sectionbegin 5. {References}
\\ \star Amlie, Kristian. Collegue, Principal Engineer (kristian.amlie@northern.tech)
\\ \star Campos, Lluis. Collegue, Senior Engineer (lluis.campos@northern.tech)

\finishpage

\sectionbegin 6. {Skillset}
\\ Languages:
\\ \star Golang, Python, C++, C, Javascript

\\ Build Systems
\\ \star Yocto,Buildroot, Autotools, CMake

\\ Cloud Technologies
\\ \star k8s

\\ Packaging
\\ \star Debian

\sectionbegin 7. {Roles and Responsibilities}

\\ \star Head of Quality Assurance, 2021--2022

\\ \star Project Lead, Customer Python deliverable, January 2019 -- May 2019

\end
