% This page defines the format for Vitas

\font\tentt=cmtt10
\font\sltt=cmsltt10
\let\mainfont=\tenrm

\font\tenssb=cmssbx10
\font\twelvess=cmss10 at 12pt
\font\ninerm=cmr9
\font\eightrm=cmr8
\font\eightsl=cmsl8
\def\sc{\eightrm}

\def\TeX{T\hbox{\hskip-.1667em\lower.424ex\hbox{E}\hskip-.125em X}}
\font\logo=manfnt % font used for the METAFONT logo
\def\MF{{\logo META}\-{\logo FONT}\spacefactor1000\relax}
\def\slMF{{\logo 89:;}\-{\logo <=>:}} % slanted version
\let\,=\thinspace
\def\?{\mkern-1mu}
\chardef\_=`\_

\def\og#1{\leavevmode\smash{\vtop{% crude approximation of Polish ogonek
  \baselineskip0pt\lineskip0pt\lineskiplimit0pt
  \ialign{##\crcr\relax#1\cr
    \hidewidth\kern.2em
    \dimen0=.0040ex \multiply\dimen0\fontdimen1\font
    \kern-.0156\dimen0`\hidewidth\cr}}}}

\def\p #1.{\vskip 3pt plus 2pt minus 1pt
 \noindent\hbox to 20pt{\hss\bf#1. }\hangindent 20pt\ignorespaces}

\outer\def\finishpage{\par\vfill\eject}

\def\\{\par\hangindent 20pt\noindent\ignorespaces}

\outer\def\sectionbegin #1. #2{\vskip 0pt plus 400pt\penalty-100
 \vskip 12pt plus -394pt minus 4pt
 \noindent\tenssb\hbox to 0pt{\hss#1. }#2
 \par\penalty10000\vskip 3pt plus 2pt minus 1pt\rm}

\def\pat#1 {\par\noindent
 \hbox to 40pt{\hss #1\hskip 5pt}\hangindent 40pt\ignorespaces}

\def\yskip {\vskip 3pt plus 2pt minus 1pt}
\def\yyskip {\penalty-100\vskip 6pt plus 4pt minus 2pt}
\def\xskip {\hskip 7pt plus 3pt minus 4pt}

\def\star{\hbox to 0pt{\hss *}}

\newdimen\unicodeptsize \unicodeptsize=10pt

\begingroup
\catcode`@=11 % we will define two private macro names

\gdef\Uni#1:#2:#3:#4:#5<#6>% ems:cols:rows:-hoff:rows+voff<hexbitmap>
 {\leavevmode \hbox to#1\unicodeptsize
    {\special{" 0 0 moveto currentpoint translate
                \unic@deptsize \unic@deptsize scale #2 #3 true
                [24 0 0 -24 #4 #5] {<#6>} imagemask}\hss}}
\gdef\NUni#1:#2:#3:#4:#5<#6>% ems:cols:rows:-hoff:rows+voff<hexbitmap>
 {\leavevmode \hbox to#1\unicodeptsize
    {\special{" 0 0 moveto currentpoint translate
                \unic@deptsize \unic@deptsize scale #2 #3 true
                [74 0 0 -74 #4 #5] {<#6>} imagemask}\hss}}

{\catcode`p=12\catcode`t=12\gdef\uni@ff#1pt{#1}}
\gdef\unic@deptsize{\expandafter\uni@ff\the\unicodeptsize\space}

\endgroup

\def\today{\ifcase\month\or
  January\or February\or March\or April\or May\or June\or
  July\or August\or September\or October\or November\or December\fi
  \space\number\day, \number\year}

%%%%%

\line{\twelvess CURRICULUM VIT\kern-1.5pt\AE\hfill\eightrm\today}
\vskip 10pt

\sectionbegin 1. {Biographical and Personal Information}
\\Ole P. Orhagen, born August 21, 1989, Gj{\o}vik, Oppland, Norway; Norwegian citizen.
\sectionbegin 2. {Academic History}
\\University of the Sunshine Coast, September 2010 -- June 2014; B.S., Major English, Minor Mathematics.
\\University of Oslo, September 2014 -- May 2019; B.S. Computer Science, M.Sc. in
Cyber\-n\-e\-tics, May 2019. Thesis: ``The RRT Funnel Algorithm.''

\sectionbegin 3. {Employment Record}
\\Teacher's Assistant, University of Oslo, Oslo, 2015--2016.
\\Intern, Northern.tech, Oslo, 2016--2019.
\\Junior Software Engineer, Northern.tech, Oslo, 2019--2021.
\\Software Engineer, Northern.tech, Oslo, 2021--2022.
\\Senior Software Engineer, Northern.tech, Oslo, 2022--

\sectionbegin 4. {Publications}

\\ \star Orhagen, Ole Petter. The development of the RRT-Funnel motion planning algorithm. MS thesis. 2019, link: {\tt https://www.duo.uio.no/bitstream/handle/10852/69325/1/olepor\_thesis\_2019.pdf}.
\\ \star Orhagen, Ole Petter, Marius Thoresen, and Kim Mathiassen. "The rapidly exploring random tree funnel algorithm." 2022 8th International Conference on Mechatronics and Robotics Engineering (ICMRE). IEEE, 2022, link: {\tt https://ieeexplore.ieee.org/abstract/document/9734089}.


\sectionbegin 5. {References}
\\ \star Amlie, Kristian. Collegue, Principal Engineer (kristian.amlie@northern.tech)
\\ \star Campos, Lluis. Collegue, Senior Engineer (lluis.campos@northern.tech)

\finishpage

\sectionbegin 6. {Skillset}
\\ Languages:
\\ \star Golang, Python, C++, C

\\ Build Systems
\\ \star Yocto, Buildroot, Autotools, CMake

\\ Cloud Technologies
\\ \star k8s

\\ Packaging
\\ \star Debian, Arch Linux

\sectionbegin 8. {Online Profiles}

\\ \star GitHub: https://github.com/oleorhagen/

\\ \star LinkedIn: https://no.linkedin.com/in/ole-petter-orhagen-ab6943187

\sectionbegin 7. {Experience}

\\ \star Northern.tech Senior Software Engineer 2021-2024

Since 2021 Ole has been working as a senior software engineer, which includes a
larger responsibility, including training interns and new employees, and has had
the main responsibility for parts of our codebase. He has had the main
responsibility for releasing our software to our end users, and has at times
maintained a backlog for the developer team. He has designed, and introduced to
the team a new software release solution, moving from semantic LTS releases, to
a rolling software release. He has also helped other developers with their
problems, and addressed internal bottlenecks in the development process. Through
this role, even though he has most experience with SCRUM, he has also experience
working with Kanban.

\vskip 12 pt

\\ \star Northern.tech Head of Q.A. 2021-2022

For a year, Ole Petter had the main responsibility for Quality Assurance at
Northern.tech. This included following up on, and ensuring that the released
software is of a high enough quality, to be able to confidently release it to
customers. As a part of the Q.A responsibility he created, kept and managed a
backlog of work and tasks, as well as gathering statistics on the quality of our
test suites. He also introduced hardware-testing, in addition the the emulated
testing which was already done.

\vskip 12 pt

\\ \star Northern.tech Software Engineer 2020-2021

In the role as a Software Engineer, Ole Petter got the responsibility of leading
and designing new functionality for the companys products. This involved
interacting with customers, learning their needs, design a solution, break it up
into tasks and implement this in collaboration with the team.

\vskip 12 pt

\\ \star MacGregor Løsningsarkitekt og Prosjektleder 01.01.2020-01.06.2020

As part of a customer deliverable, Ole Petter had the responsibility as project
lead, and solutions architect for a Python client, designed to run on the
customers ships for long periods of time, and often in a fully isolated
environment. With three developers on the team, and a short time horizon, the
project had the customer in the loop the whole time, and provided an iterated
development model, including implementing a full CI/CD solution, and creating a
release methodology, designing of specifications, and verification of the given
software against these specifications. The method of work was Kanban, and the
team operated independently of the rest of the company. This taught Ole Petter
to manage a project, including both the social aspect, as well as the
proffessional interaction with a customer, and understanding fully the technical
limitations of the given product.

\vskip 12 pt

\\ \star Northern.tech Junior Software Engineer 2018-2020

In the role as a Junior Software Engineer, Ole Petter implemented new
functionality in existing products. He implemented already created tasks and
tickets into functionality. Through this period he learnt new things every day,
and taught him to work systematic and structured, and hence made him comfortable
working in a team, and in the SCRUM methodology.

\vskip 12 pt

\\ \star Norwegian Defence Research Institute Research assistant 01.08.2019 - 01.01.2021

After Ole Petter finished his M.Sc. he got a part-time contract as a research
assitant, with the goal of creating a research paper, together with his former
advisor, and his doctorates student. This cooperation resulted in a published
research paper, which built on the work done during his master's thesis. This
experience helped him communicate his work, break it down into structured
pieces, and teach this to other people. It also taught him to follow strictly
defined processes, such as the process surrounding research paper publishing,
which is thorough, and requires a lot of attention to detail.

\vskip 12 pt

\\ \star Northern.tech Software Intern 2016-2019

In the role as a Software Intern, Ole Petter worked part-time (40\%), and
through this got the opportunity to explore and learn multiple aspects of
software engineering, while still in school. He got to do a large array of work,
ranging from backend, to frontend and embedded development, Q.A, and maintaining
the company's legacy product. This gave him a lot of motivation, and taught him
to view the products from many different angles, which have been very valuable
later in his career.

\end
