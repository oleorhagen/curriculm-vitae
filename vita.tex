\documentclass[11pt]{article}
\usepackage[utf8]{inputenc}
\usepackage[T1]{fontenc}
\usepackage{graphicx}
\usepackage{longtable}
\usepackage{wrapfig}
\usepackage{rotating}
\usepackage[normalem]{ulem}
\usepackage{amsmath}
\usepackage{amssymb}
\newtheorem{theorem}{Theorem}
\usepackage{capt-of}
\usepackage{hyperref}
\usepackage{tikz}
\usetikzlibrary{positioning}
\usetikzlibrary{trees}
\usepackage{soul}
\usepackage{csquotes} % \begin{displayquote}
\date{\today}
\title{Curriculum Vitae}
\hypersetup{
  pdfauthor={},
  pdftitle={},
  pdfkeywords={},
  pdfsubject={},
  pdfcreator={Emacs},
  pdflang={English}}

%
% arg1 - employer
% arg2 - position
% arg3 - timeline
\newenvironment*{experience}[3]{
  \par\noindent\rule{\textwidth}{0.4pt}
  #1 \\
  #2 \\
  #3 \\
  \par\noindent\rule{\textwidth}{0.4pt}
}
{\par\noindent\rule{\textwidth}{0.4pt}} % enddef empty for now


\begin{document}

\section{Bio}

\section{Nøkkelferdigheter}

\section{Arbeids og Prosjekterfaring}

\begin{experience}{Northern.tech}{Senior Software Engineer}{2021-2024}
  Ole Petter har siden 2021 jobbet som seniorutvikler, hvilket medfører større
  ansvar. Blant annet har han vært mentor for nyansatte, og hatt hovedansvaret
  for deler av kodebasen. Han har hatt ansvaret for å release software til
  kundene og holde en backlog vedlike for utvikler-teamet. Han har innført ny
  automatisk software-release-metodikk og hjulpet andre utviklere i hverdagen
  deres igjennom å adressere interne flaskehalser i utviklingsprosessen. Gjennom
  denne rollen fikk han også erfaring med andre arbeidsmetoder, som Kanban.
\end{experience}

\begin{experience}{Northern.tech}{Head of Q.A.}{2021-2022}
  Ole Petter hadde i ett år hovedansvaret for testing hos Northern.tech. Dette innebar å følge opp og sikre at
  produktene oppfyller høye kvalitetsstandarder før de ble levert til kundene. Som en del av QA-ansvaret organiserte
  han en egen backlog og samlet statistikk over utført arbeid. Han introduserte også hardware-testing, i tillegg til den
  eksisterende emulerte testingen som var gjort fra før.
\end{experience}

\begin{experience}{Northern.tech}{Software Engineer}{2020-2021}

  I rollen som Software Engineer fikk Ole Petter ansvaret for å lede og designe ny
  funksjonalitet for selskapets produkter. Dette involverte omfattende
  undersøkelser for å forstå kundenes behov, designprosesser, oppgaveinndeling og
  implementering i samarbeid med teamet.

\end{experience}

\begin{experience}{MacGregor}{Løsningsarkitekt og Prosjektleder}{01.01.2020-01.06.2020}
Beskrivelse
MacGregor er en global leverandør av ingeniørløsninger for marinesektoren.
Prosjektet involverte utviklingen av en oppdateringsklient som skulle integreres med en eksisterende OTA-løsning i
Python, spesifikt laget for PowerPC-plattformen. Den utviklede løsningen måtte være robust og konstruert for å
fungere på skip over lengre perioder uten tilsyn.
Med tre utviklere og en tidshorisont på seks måneder ble prosjektet nøye overvåket av kunden, som var aktivt
involvert gjennom hele utviklingsprosessen. Dette inkluderte testing, kravspesifikasjon og leveransemetodikk.
Arbeidsmetoden som ble benyttet var Kanban, og teamet opererte uavhengig av resten av selskapet.

Ansvar

Ole Petter fungerte som prosjektleder og løsningsarkitekt. Han hadde ansvaret for designet av løsningen, brøt ned
oppgavene og samarbeidet aktivt med utviklingsteamet gjennom hele prosessen. Produktet gjennomgikk
automatisk testing i flere stadier, inkludert kundens hardware, som en integrert del av CI/CD-prosessen for
prosjektet.

\end{experience}

\begin{experience}{Northern.tech}{Junior Software Engineer}{2018-2020}
  Beskrivelse
  I rollen som Jr. Software Engineer utførte Ole Petter betydelig arbeid ved å implementere ny funksjonalitet i
  eksisterende produkter. Han fullførte også oppgaver som allerede var nedbrutt og tildelt av lagets leder. Denne
  erfaringen utviklet hans evne til å arbeide systematisk og målrettet, og samtidig ble han komfortabel med å jobbe i
  team under SCRUM-metodologien.

\end{experience}

\begin{experience}{Forsvarets Forskningsinstitutt}{Midlertidig ansatt}{01.08.2019-01.01.2020}
  Beskrivelse
  Etter fullføringen av Ole Petters mastergrad fikk han en deltidskontrakt ved Forsvarets forskningsinstitutt. Her
  fortsatte han oppfølgingen av masteroppgaven sin i samarbeid med en professor og Ph.D.-studenter. Resultatet av
  dette samarbeidet var en ny forskningsartikkel som bygde videre på Ole Petters arbeid. Denne erfaringen bidro til å
  utvikle hans evne til å jobbe strukturert og følge klart definerte prosesser, spesielt i forbindelse med
  artikkel-levering og -publisering.
\end{experience}

\begin{experience}{Northern.tech}{Software Intern}{2016-2019}
  Beskrivelse
  I rollen som Software Intern, jobbet Ole Petter deltid (40%) og fikk anledning til å utforske flere ulike aspekter av
  selskapet. Han var involvert i en rekke oppgaver, inkludert QA, testing, vedlikehold av selskapets legacy-produkt
  (CFEngine), til utvikling av ny funksjonalitet på det eksisterende produktet (Mender).
\end{experience}

\section{Kurs}

\section{Utdanning}

\end{document}
